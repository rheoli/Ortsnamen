% Geschichtsaufsatz zum Thema:
% Keltische Ortsnamen in der Schweiz und ihre Entwicklung
% Datum:  03.02.1996

%%%%%  Hauptdefinitionen f"ur Texte  %%%%%%
\documentclass[10pt]{article}
\usepackage{latexsym}   % ?????
\usepackage{german}     % Deutsche Sonderzeichen und Befehle
\usepackage{a4}         % A4 Ausrichtungen

%% \pagestyle{headings}    % Seitenaufbau mit Kapitelnamen als Kopfzeile
\pagestyle{myheadings}
\markright{\protect\rule[-5pt]{\linewidth}{1pt}\hspace{-\linewidth}%
{\protect\large Geschichte\ \  Keltische Ortsnamen in der Schweiz}%
}

\setlength{\evensidemargin}{-1.2cm}   % Breite f"ur l"ocherseitem Rand einstellen
\setlength{\oddsidemargin}{-0.5cm}    % Breite von gegenseitigem Rand

\setlength{\topmargin}{-1cm}          % H"ohe von oberen Rand einstellen

\setlength{\textwidth}{17.5cm}        % Textfensterbreite
\setlength{\textheight}{23.5cm}       % Textfensterh"ohe
% \setlength{\mathindent}{35mm}       % Formeln Abstand vom Rand

\setlength{\parindent}{0em}           % Absatzeinr"uckung

\renewcommand{\baselinestretch}{1.2}  % ???

%%%%%%%%%%%%%%%%%%%%%%%  Dokument Beginn  %%%%%%%%%%%%%%%%%%%
\begin{document}

%%===============  Titelbild  =================%%
\begin{titlepage}
\begin{center}
\vspace*{30mm}
{\huge \bf Keltische Ortsnamen}\vspace{8mm}\\

{\huge \bf in der Schweiz}\vspace{8mm}\\

{\LARGE \bf und ihre Entwicklung}\vspace{60mm}\\

{\Large Geschichte 3. Semester}\vspace{6mm}\\

{\large im Februar 1996}\vspace{10mm}\\

{\large von}\vspace{6mm}\\
{\Large St. Toggweiler}\vspace{6mm}\\
{\large Klasse IA94}\vspace{10mm}\\

{\large Dozent: Dr. R. K"aser}\vspace{8mm}\\
{\large HTL Brugg-Windisch}

\end{center}
\end{titlepage}
%%------------------  Ende Titelbild  --------------%%


%%=================  Inhaltsverzeichnis  ============%%
\tableofcontents
\newpage
%%-------------  Ende Inhaltsverzeichnis  ----------%%


%%================== Beginn der Texte  =============%%

\setlength{\parskip}{0.6em}           % Absatzabst"ande nur f"ur Texte

\section*{Einleitung}
Ich hatte mir die Aufgabe gestellt, Ortsnamen der Schweiz zu suchen, die
keltischen Ursprungs sind. Es ist jedoch schwierig ohne die Geschichte
der Kelten, die Entwicklung der Ortsnamen zu verstehen, da die Ortsnamen ausserdem
von den Germanen beeinflusst wurden. Von der keltischen Kultur ist in der
Schweiz nicht mehr viel "ubrig geblieben. Die archeologischen Funde 
beziehen sich meist auf H"ugelgr"aber und ihren Inhalten. Einzig in
Irland sind heute noch Br"auche von dieser Zeit intakt. Schriften von
den Kelten in der Schweiz sind sehr rar, im ganzen sind es drei
Schriftst"ucke die beschriftet mit der keltischen Ogham-Schrift\footnote{Bis heute
konnte noch nicht alle Zeichen entschl"usselt werden.} sind. Um die Jahrhuntertwende
hatte die Keltenforschung Hochkonjuktur, es wurde sehr viel geschrieben,
auch solches, das heute nach neusten Kenntnissen nicht mehr stimmen kann.

%% \twocolumn
\section{Geschichte der Kelten}
Um sprachwissenschaftlich die keltischen Ortsnamen zu deuten, ist es n"otig,
etwas "uber die Entwicklung des Keltischen zu erfahren. Der Ursprung der 
Sprachen der heutigen Bev"olkerung in Westeuropa sind alle Abk"ommlinge des
Indogermanischen oder Indoeurop"aischen\footnote{Das Baskische bildet eine
Ausnahme, deren Ursprung man bis heute noch nicht kennt}.

\subsection{Die Indogermanen}
Der Ursprung der Indogermanen wird in den Steppen vom Kaukasus vermutet, dies
wurde anhand des Wortschatzes ermitteln. Man muss jedoch
auch miteinrechnen, dass sich das Klima seit dieser Zeit stark ver"andert hat.\\
Die Urindogermanen\footnote{Die Bezeichnug f"ur die Zeit, in der noch alle
Indogermanen sich verstanden hatten, sich also nur durch Dialekte
unterschieden} lebten in einem feuchten Klima den sie kannten B"aume wie
die Buche oder Eiche. Sie hatten auch sehr fr"uh das Arbeitspferd gekannt.
Die Entdeckung des Pferdes als Arbeitstier wird von den Geschichtsforschern
als sehr wichtig erachtet, den dadurch konnten sich die Menschen viel schneller
fortbewegen. Man musste zuvor Ochsen f"ur diese Arbeiten benutzen die 
viel langsamer waren.\footnote{Nach dem Text von Wolfgang Meid \cite{IDG:MEID}}\\
Die Grundsprache, das Urindogermanische konnte vor allem durch Vergleiche
mit den Indogermanischen Sprachen Altgriechisch, Altpersisch (Arisch) und
Sanskrit (Altindisch) konstruiert werden. Weshalb die Urindogermanen
aus ihrer Heimat auswanderten ist bis heute nicht bekannt. Es ist aber sehr
wahrscheinlich, dass es mit der Zeit Platzprobleme gab oder auch Streit
unter den verschiedenen Sippen. Wann genau die V"olkerwanderung begann kann
nicht genau gesagt werden, da es keine archeologischen Fundamente gibt, die
auf dies hinweisen. Es kann nur anhand der Lautverschiebungen (siehe Abschnitt
\ref{Laut}) vermutet werden,
indem man die alten Schriften vergleicht.\\
Die Indogermanischen Sprachen werden in zwei Hauptgruppen aufgeteilt; die Centum-
und Satem-Sprache.
Diese Bezeichnung ist entstanden beim betrachten der Zahl 100 der verschiedenen
Indogermanischen Sprachen. Zu den Centum-Sprachen geh"ort z.B. das Keltische, 
Germanische, Italische, zu den Satem-Sprachen z.B. Altindische, arische
Sprachen\footnote{wie z.B. Altpersisch, Iranisch}

\subsection{Die Kelten}
Von den Kelten sind in der Schweiz nur sehr wenige Schriftst"ucke bekannt,
wie bereits oben erw"ahnt,
die die gefunden wurden, beinhalten jedoch meist nur eine Auswahl von
Namen (Vornamen, ON, \dots). Sie wurden in der keltischen Ogham-Schrift
geschrieben, die teilweise noch nicht entziffert werden konnte.\\
Die keltische Sprache wird heute in zwei Gruppen aufgeteilt:
\begin{itemize}
  \item {\bf Festlandkeltisch}: Gallisch, Ibero-Keltisch
  \item {\bf Inselkeltisch}: Irish, Gaelisch, Kymrisch\footnote{Walisisch}
\end{itemize}
Das Festlandkeltisch\footnote{Das Bretonische in Frankreich ist ein Abk"ommling des Cornischen in S"ud-West
England} ist nach schon seit l"angerer Zeit ausgestorben, um ca. 600 n.Chr.
nach dem Einfall der Germanen in keltische Gebiete. Das Festlandkeltisch kann nur durch Ableiten aus dem Inselkeltisch
und unseren Ortnamen (Flurnamen, \dots) hergeleitet werden. Auch Texte von
Julius Caesar und von einigen Griechen k"onnen behilflich dazu sein (siehe
auch Abschnitt \ref{Wissen}).\\
Durch die R"omischen Eroberungen des Gallischen Gebietes wurde die keltische
Kultur immer mehr verdr"angt. Es gab jedoch auch eine Durchmischung der 
beiden Kulturen. Der Zivilisationsstand der Kelten entsprach wahrscheinlich nicht
ganz dem der R"omer, jedoch im Strassenbau und im Kriegerischen
standen sie den R"omer in nichts nach. Es wird vermutet, dass viele
sogenannte ''R"omische Strassen'' bereits von den Kelten gebaut wurden. 

\subsection{Die Illyrer}
Die Illyrer hatten ihr Domizil im schweizerischen, "ostereichischen Alpengebiet
sie werden auch einige unserer Ortsnamen\footnote{Es sollte hier nicht nur
von Ortsnamen die reden sein, es geht nat"urlich auch um Flur-, Fluss- und
Gebietsnamen usw.} beeinflusst haben. Ueber sie wird jedoch nicht sehr viel
geschrieben. Sie sind ein geheimnisvolles Volk, dass keine Schriften hinterlassen
hat, jedoch vieles beeinflusst haben k"onnte.

\subsection{Die Germanen}
Die Letzen, die die keltischen Ortsnamen beeinflussten waren die Germanen,
die in unserem Gebiet mit dem Namen Alemannen bezeichnet werden.
Als Nichtkelten hatten sie den gr"ossten Einfluss auf die keltischen
Ortsnamen. Obwohl sie auch indogermanischer Abstammung sind, kann man zwischen
dem Keltischen und Germanischen eigentlich keine Aehnlichkeiten mehr finden, weshalb
Ortsnamen auch ganz anders Intepretiert worden sind (siehe Abschnitt~\ref{Etym}).


\section{Lautverschiebung / Etymologische Vergleiche}
Die gesprochene Sprache wird laufend ver"andert. Aus Bequemlichkeit werden
z.B. Buchstaben weggelassen oder Laute abgeschw"acht. Heute wurde dieser
Vorgang ein bisschen gebremst, indem man eine Norm "uber eine Sprache
definiert, damit benutzen alle die gleiche Grundvoraussetzug, wenn sie
die Sprache erlernen.

\subsection{Die Lautverschiebung} \label{Laut}
Ein Problem bei der Zur"uckverfolgung der Ortsnamen sind die Lautverschiebungen,
bei den keltischen Ortsnamen, in der Schweiz, ist vor allem die 2.~Lautverschiebung
der Germanen wichtig. Die 
1. Lautverschiebung der Germanen ist etwa mit der der Kelten zeitlich zusammengefallen.
Ein Beispiel dazu ist der Rhein der Indogermanisch *Reinus\footnote{Der *
vor einem Wort bedeutet, dass dieses Wort ein konstuiertes Wort ist, also
abgeleitet aus verschiedenen Quellen} geheissen hat. Die keltische Verschiebung
ei $>$ e machte daraus *R\=enos und die der Germanischen ei $>$ i *R\=in\footnote{
siehe dazu Hans Krahe \cite{IDG:KRAHE} S. 95-96}. Diese 1.~Lautverschiebung war
ca. vor 2000 v. Chr., zu dieser Zeit gab es noch keine
Germanen in unserem Gebiet\footnote{Mit {\sl unserem} Gebiet bezeichne ich immer
die Schweiz (Deutschschweiz)}. Der grosse Einfall der Germanen in die Schweiz
begann, nachdem die r"omischen Truppen abgezogen wurden, um das Kerngebiet des r"omischen Reiches
am zusammenbrechen zu hindern, was jedoch nicht erreicht wurde, das war um 500 n.Chr. Jedoch das Zusammenwachsen der
beiden Kulturen, Gallo-R"omer und Germanen, ging sehr langsam vor sich, es
gab also einen grossen Kulturellen austausch.

\onecolumn
\subsection{Die Etymologie und die Wortwandlung} \label{Etym}
Eine Lautverschiebung wirkt sich immer auf alle gesprochenen W"orter aus, mit
Ausnahmen, es ist also wenn man die Verschiebung kennt m"oglich das urspr"ungliche
Wort herzuleiten. Wenn aber ein Gebiet von einem anderen Volk, mit anderer
Sprache erobert wird, werden meist die alten Orts-, Fluss- und Bergnamen beibehalten.
Da die Eroberer eine andere Sprache sprechen versuchen diese die Namen ihrer eigenen
Sprache anzupassen, das hat zur Folge, dass der Sinn der Namen ge"andert
wird (siehe dazu Anschnitt~\ref{Winterthur}). Diese Ver"anderungen
k"onnen f"ur die Namensforschung sehr schlecht sein, da damit die Namen
nicht mehr richtig hergeleitet werden k"onnen. Besonders dann, wenn die 
schriftlichen Quellen nur bis ins Mittelalter reichen.

\subsection{Die Quellen der Namensforscher}
In der Namensforschung muss man sich vielfach auf griechische oder r"omische 
Schriftquellen verlassen, obwohl man die genaue Lautwiedergabe dieser beiden Sprachen
nicht kennt. Ein weiterer Unsicherheitsfaktor ist, woher der Schreiber
diese W"orter geh"ort hat. Wenn er im Gebiet selber war k"onnte er die W"orter falsch verstanden
haben (im Wortlaut), da er eine andere Sprache spricht und es dadurch falsch wiedergab.
Wiederum ist es m"oglich, dass er die W"orter nur von einem Reisenden
aus diesem Gebiet geh"ort hat und dieser dem Schreiber die Laute falsch "ubergab.\\
Vielfach werden auch auf die Kriegstageb"ucher von Julius Caesar verwiesen.
Er hat alle geh"orten W"orter und Landschaftsbezeichnungen aufgeschrieben.
Wie seri"os er sich jedoch an die Wortlaute gehalten hat, ist heute nicht
mehr feststellbar. Am besten man versucht einen Durchschnitt aus allen
Quellen zu finden, dadurch wird die Wahrscheinlichkeit am gr"ossten, dass
man den wahren Wortlaut treffen kann.

\section{Beispiele von keltischen Ortsnamen der Schweiz}
\subsection{Winterthur - *Vitodurum} \label{Winterthur}
Beim Ortsnamen Winterthur\footnote{Eine Zusammenfassung aus der Festschrift
zum 75.~Geburtstag von Hans Kl"aui \cite{WTH:KLAEUI} S.53-55} fallen die Probleme an,
dass die heutige Bezeichnung {\sl Winterthur} nicht direkt auf das keltische
*Vitodurum zur"uckzuf"uhren ist, bereits die Betonung war ganz anders gewesen
*Vit\'udurum (also ein langes u). Bei der 2. Lautverschiebung der
Alemannen\footnote{Nachfolger der Germanen die sich in unserem Gebiet
niedergelassen hatten.} m"usste ein Lautwandel des {\sl t} in ein {\sl z}
oder {\sl ss} erfolgt worden sein\footnote{als Beispiel engl.
wa\underline{t}er wurde zu deutsch Wa\underline{ss}er gewandelt.}.
Da jedoch zu dieser Zeit, um 700 n.Chr. noch mehrheitlich die B"ucher und
Chroniken in Lateinisch geschrieben wurden, "ubernahm man einfach den
lateinischen Lautausdruck und nicht der der Germanen mit der Wandlung.
Es ist zudem m"oglich, dass die Alemannen den Ortsnamen als *Vidoduro geh"ort
hatten und es daher keine Lautverschiebung gab.

\subsubsection{Wortentwicklung des Namen Winterthur}
Nach verschiedenen Urkunden:\\
{\sl kelt. *Vitodurum $>$ Vituduro/Vitudoro um 280 $>$ *Vidoduro um 500 $>$
alem. *Witoturo um 700 $>$ Venterdura 843 $>$ Wintarduro 856 $>$
Winturdura 865 $>$ Winterdura 883 $>$ Winterthura 1155 $>$ Winterthur 1273}\\
Wie man bei dieser Entwicklung sieht, ist die Abh"angigkeit vom Schreiber sehr
gross, denn alle diese Namen wurden von anderen Quellen entnommen. Interessant
ist, dass innerhalb von 40 Jahren (843-883 n.Chr.) der heutige
Wortlaut von Winterthur entstand.\\
Wie sich das Vito- nach Winter- ver"andern konnte, kann heute nur vermutet
werden:
\begin{itemize}
  \item Es k"onnte ein alemannischer Name gewesen sein, wie {\sl Winithere} oder
        {\sl Winithari}, der ein Oberhaupt dieser Siedlung war.
  \item Die Aussprache von *Vido- k"onnte auch als Wintar verstanden worden
        sein, was Altdeutsch {\sl Winter} bedeutet. Die Wandlung von -tar-
        nach -tur- ist eine kleine Wortspielerei vom Mittelalter.
\end{itemize}
Die Endung von *Vitudurum {\sl -durum $>$ -duro $>$ -dura} wurde wahrscheinlich
vom nahegelegenen Fluss Thur beeinflusst, der in keltor"omischer Zeit *Dura
oder *Duria hiess. Die Endung muss also die gleiche Lautentwicklung
durchgemacht haben, wie die des Flusses\footnote{idg. *dhur\=a (=Flusslauf) $>$ kelt. *Dura
$>$ T\^ura 870 $>$ Thure 1282 $>$ Tur 1324 $>$ Thur (heute)}.

\subsubsection{Die keltische Bedeutung von Vitodurum}
Das Wort *Vitodurum ist aus zwei einzelnen zusammengesetzt, Vito und Durum.
{\sl Vito} kann ein keltischer Name sein, oder es war die Bezeichnung vom
deutschen Wort {\sl Weide} (kelt. witua). {\sl Durum}\footnote{Weitere
Siedlungen mit Durum Bezeichnung sind Salodurum (Solothurn) oder *Ollodunum
(Olten), -dunum ist eine Variation von -durum} bezeichnet eine Burg
(befestigte Siedlung). Zusammengesetzt k"onnte *Vitodurum also die {\sl Burg
des Vitus} oder {\sl Weideburg} heissen. 

\subsection{Stadt Biel}
Das Stadtwappen\footnote{Zusammenfassung aus Paul Zinsli \cite{ON:ZINSLI}} mit der Axt wurde erst sp"ater eingef"uhrt, hat jedoch
nichts mit der eigentlichen Bedeutung des Namens {\sl Biel} zu tun (gilt nur
f"ur die Stadt Biel). Nach Urkunden aus dem Lateinischen hiess es Belna 
(apud Belnam) 1142. Die franz"osische Parallelform Bienne wird von der
latainisch, deutschsprachigen Nennung {\sl de Bielno} von 1179 abgeleitet sein.
Im 16.Jahrhundert wurde {\sl ze Bielne} geschrieben. Aus diesen Ableitungen
wurde geschlossen, dass der Name Biel von dem keltischen Wort *Belena
abstammen muss.
Belenus/Belinus war der Name einer alten Quellengottheit. Denselben
Grundstamm haben noch einige andere Ort- und Bezirknamen wie
Beaune (Frankreich), Beune (S"udfrankreich), le Bainoz (im freiburgischen).
Dieses Biel hat jedoch nichts mit den anderen Bieli, Bielen und Biel
anderswo zu tun.

\section{Ueber die keltische Sprachwissenschaft} \label{Wissen}
Die ersten Sprachwissenschaftler, die sich mit der keltischen Sprache befassten,
waren die Griechen wie Plinius um 2000 v.Chr., sie beschrieben die
Wortlaute, wie sie es geh"ort hatten.\\
Julius Caesar schrieb auch einiges "uber die Bezeichnungen der Kelten in seine
Tageb"ucher. Er war auch der, der den verschiedenen St"ammen Namen gab, ob
er sie "ubernommen hat oder neu erfunden sei dahingestellt.\\
Die Kunst des Schreibens begann bei den Kelten erst um 700 n.Chr. in Ogham.
Vorher muss man sich auf Quellen von anderen V"olkern halten. Die Wissenschaften
der Kelten wurde immer von den Druiden\footnote{Gelehrter, der ein sehr grosses
Wissen "uber viele Dinge hatte, er war das Pendent zum Medizinmann der Indianer}
an ihre Sch"uler "ubertragen, der sehr lange
bei seinem Meister lernen musste. Das gemeine Volk wurde von den Wissenschaften
ausgeschlossen.\\
Das grosse Forschen nach den keltischen Urspr"ungen begann er im 19. und 20.
Jahrhuntert. Zu dieser Zeit begann man auch das Altkeltische zu rekonstruieren,
die bis heute noch richtig abgeschlossen ist.

\subsection{Sprachenzuordung eines Wortes}
In den B"uchern, die ich gelesen habe\footnote{Wie \cite{IDG:KRAHE},
\cite{VOX:3}, \cite{VOX:10}, \cite{ORT:KLAEUI}, \cite{KELT:HOPFNER}},
waren vielfach Einseitig verfasst, was aber von menschlicher Natur ist.
Wir haben den Hang sehr schnell Fan von etwas zu werden.\\
Ein Beispiel dazu ist, die schriftliche Auseinandersetzung\footnote{In den
beiden B"uchern \cite{VOX:3} und \cite{VOX:10}} zwischen
J.U. Hubschmied und J. Pokorny. Hubschmied leitete sehr viele
W"orter, wie Ortsnamen, auf das Keltische zur"uck, Pokorny dagegen ist ein
Anh"anger der Illyrer. Pokorny hat ein ganzes Buch\footnote{siehe
\cite{VOX:10}} verwendet um Herleitungen von Hubschmied zu
widerlegen. Solche Auseinandersetzungen k"onnen auch sehr fruchtbar sein,
da sie zum denken anregen, denn ohne Behauptung kann man auch keinen 
Beweis suchen, nur so kommt man in einer unsicheren Wissenschaft einen
Schritt weiter.\\
Das manche hergeleiteten Namen sehr unsicher sind sieht man manchmal daran,
dass sich verschiedene B"ucher widersprechen.

\subsection{Was ist ein keltisches Wort ?}
Ein heute als keltisches anerkannter Ortsnamen muss nicht unbedingt von
den Kelten eingef"uhrt (erfunden) worden sein, denn vor den Kelten lebten
bereits andere mit einer anderen Sprache in diesem Gebiet, die auch
bereits Orten Namen vergeben hatten (haben mussten). Daher k"onnte es auch
einen gleichen Uebernahmeprozess gegeben haben, wie beim Uebergang von
den Kelten zu den Germanen (siehe Abschnitt~\ref{Etym}).\\
Von keltischer Herkunft k"onnen die Namen angesehen werden, die
indirekt die Schlange beschreiben, wie es bei Fl"ussen vielfach vorkommt
\footnote{Die Schlange wird von den Kelten nie direkt bezeichnet, sondern
immer mit einem Uebernamen versehen, wie die Wilde, usw.}, denn manche Zeichnungen
von keltischen B"uchern enthalten Zeichnungen mit Schlangensymbolen (siehe
Titelblatt).

\section*{Schlussfolgerung/Schlussbemerkung}
Das Gebiet "uber die Namensforschung ist sehr interessant, es nimmt jedoch
sehr viel Zeit in Anspruch, besonders wenn man sich nur auf eine bestimmte
Kulturgruppe bezieht. Auch die Suche nach einem geeigneten Buch ist sehr
schwierig, z.T. werden die Herleitungen sehr kurz gehalten.
Es ist daher wichtig, dass man mehrere B"ucher hat, damit man vergleichen
kann und eventuelle Fehler erkennen kann. Vielfach werden auch sehr grosse
Vorkentnisse vorausgesetzt. Was heute fehlt, ist ein Buch, das sich mit der
schweizerischen, keltischen Ortsnamen befasst, dass alle bisherigen und
erkannten Fehler ber"ucksichtigt.


%% Bibliography
\newpage
\bibliography{Ortsnamen}
\bibliographystyle{plain} 

%%-------------------  Ende des Textes  ------------------%%

\end{document}
