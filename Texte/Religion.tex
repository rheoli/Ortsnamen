% Thema: Geschichte
% Titel: Vergleiche von Religionen
% Datum: im 4. Sem.

%%%%%  Hauptdefinitionen f"ur Texte  %%%%%%
\documentclass[11pt,twoside]{article}
\usepackage{latexsym}   % ?????
\usepackage{german}     % Deutsche Sonderzeichen und Befehle
\usepackage{fancyheadings}
\usepackage{rotating}
\usepackage{epsfig}     % Package um EPS-Figuren einzubinden
\usepackage{a4}         % A4 Ausrichtungen

\setlength{\evensidemargin}{-1.2cm}   % Breite f"ur l"ocherseitem Rand einstellen
\setlength{\oddsidemargin}{-0.5cm}    % Breite von gegenseitigem Rand

\setlength{\topmargin}{-1cm}          % H"ohe von oberen Rand einstellen

\setlength{\textwidth}{17.5cm}        % Textfensterbreite
\setlength{\textheight}{23.5cm}       % Textfensterh"ohe
% \setlength{\mathindent}{35mm}       % Formeln Abstand vom Rand

\setlength{\parindent}{0em}           % Absatzeinr"uckung

\renewcommand{\baselinestretch}{1.2}  % ???

\pagestyle{fancy}
\lhead[\thepage]{Vergleiche von Religionen}
\chead{}
\rhead[Vergleiche von Religionen]{\thepage}
\lfoot{}
\cfoot{}
\rfoot{}
\setlength{\headrulewidth}{1.0pt}
\setlength{\footrulewidth}{0pt}


%%%%%%%%%%%%%%%%%%%%%%%  Dokument Beginn  %%%%%%%%%%%%%%%%%%%
\begin{document}

\setlength{\parskip}{0.6em}           % Absatzabst"ande nur f"ur Texte

\section*{Die germanische Religion im Vergleich zu anderen Religionen}
Die germanische Religion wurde in den Jahren 800-1200 n.Chr. allm"ahlich in
ihrem ganzen vorherrschenden Gebiet durch das Christentum ersetzt. Somit ist
es der germanischen Religion wie allen anderen indogermanischen Religionen
ergangen, sie wurden durch eine (neue) andere ersetzt. Es waren jedoch nicht
immer die gleichen Religionen die diese ersetzen. In Europa war es vor allem
das Christentum, jedoch im Nahen Osten (Iran, T"urkei) wurde sie vom Islam
verdr"angt und in Indien durch den Buddhismus. Jedoch wird sich in allen
abl"osenden Religionen noch kleine Br"auche finden, die auf die alte Religion
weisen.\\

Wie die germanische Religion vertrieben wurde kann heute nicht mehr genau
ermittelt werden. In Island wurde der christliche Glaube in sehr kleinen
Schritten eingef"uhrt, daraus kann geschlossen werden, das er sehr friedlich
von der einen Religion zur anderen "ubergef"uhrt wurde, aber im Vergleich zu
den Missionaren auf dem afrikanischen Kontinent, die mit sehr viel Gewalt den
neuen Glauben einf"uhren, k"onnte man sich ein Bild machen, wie es bei uns
war. Durch einen Religions"anderung kann das ganze gesellschaftliche System
zusammenbrechen. Die Religion kann auch sehr gut als Machtmittel gebraucht
werden. Wie k"onnte man sonst eine so grosse Masse wie in Indien ohne grosse
Milit"arpr"asents unter Kontrolle halten. Dadurch, dass die Religion
verspricht, wenn man sich in diesem Leben gut benimmt kommt man im n"achsten
als besser betuchter Mensch zur Welt, dass gibt den Menschen halt in ihrer
verzweifelten Lage. Daher ist auch die Religion in den "armeren Schichten der
Menschen mehr verankert.\\

Bei alten, ausgestorbenen Religionen ist meist nicht sehr viel bekannt,
da das Niederschreiben der Geschichten nicht Brauch war oder man die Schrift
noch gar nicht kannte. Viele Niederschriften von alten Religionen stammen
daher erst aus der Zeit, als Europa bereits zum gr"ossten Teil Christianisiert
war, also etwa um das 12.Jh. n.Chr. Dazu geh�ren die germanische Edda wie
auch die Geschichte von Merlin.\\
Um die Christianisierung in Europa zu beschleunigen wurden alte Br"ache
(Riten) "ubernommen und unter neuem Namen, meistens einem Heiligen zugewiesen,
weiterzelebriert. Es wurden auch Menhire (stehende Steine) aus der Steinzeit
mit christlichen Ornamenten versehen, um so die Gl"aubigen auf die neue
Religion zu bringen. Alte Begriffe wurden im Ausdruck beibehalten, jedoch
inhaltlich ver"andert, z.B. Gott, Himmel, Erde, heilig u.dgl.\\

F"ur die Religionswissenschaftler w"are es einfach alle
indogermanischst"ammigen Religionen zusammenzunehmen und damit eine gesamt
Religion zu erstellen, mit der dann die L"ucken der einzelnen Religionen
gef"ullt werden k"onnen. Dies ist jedoch nicht sehr einfach und zudem noch
gef"ahrlich, da durch die V"olkerwanderung um das 2. Jt. v.Chr. sich die
indogermanischen St"amme ausbreiteten und neue L"andereien eroberten. Bei
dieser Expansion wurden die alten eingesessenen St"amme "uberlagert, aber es
ist nicht anzunehmen, dass sich nur die idg. Religion durchgesetzt hat. Damit
kamen immer neu und fremde Eigenschaften in die Religion und somit kann sie
nicht mit einer anderen gleichgesetzt werden.\\

In der germanischen Religion kann dieses Vermischen durch die beiden
G"otterst"amme, die Asen und die Wanen, dargestellt werden. Die Asen w"aren
demnach der indogermanischen Stamm, die in das Gebiet der Wanen eingewandert
sind. Der sp"ater ausbrechende Krieg w"arde damit das ethnische Problem
zwischen den beiden V"olkern aufzeichnen bei deren Vermischung. F"ur diese
Darstellung ist jedoch anzunehmen, dass die idg. St"amme, also die Asen, ein
sehr kampfs"uchtiges Volk und ihre Strukturen Patriarchal ausgebaut waren.
Im Gegensatz dazu die ureurop"aischen Einwohner eine auf Harmonie aufbauende
Gesellschaft, die Matrifokal regiert wurde.\\

Die germanische Religion ist am n"achsten mit der Keltischen verwandt,
jedoch im Vergleich ist die germanische Sprache eher mit der Baltischen oder
Slavischen verwandt. Die germanische und die keltische Rassen waren einander
"ahnlich , sie wurden von den r"omischen Geschichtsschreiber meist als gross,
meist blond und als barbarisch bezeichnet. Die einzige wichtige Differenz
dieser beiden V"olker war, dass die Kelten Druiden f"ur die religi"ose Riten
hatten und die Germanen nicht. Somit k"onnen die verschiedenen St"amme in
Europa nur nach diesem Kriterium aufgeteilt werden.\\
Eine der gr"ossten religi"osen Gemeinsamkeiten ist die Gottheit (germ.) Wotan
oder Odin bzw. (kelt) Lug. Beide wurden am Weltbaum aufgeh"angt. Sie k"ampften
beide nur mit einem Auge, wobei Odin es verkauft hatte f"ur den Quell der
Weisheit unter dem Weltbaum (Yggdrasil), jedoch Lug hatte noch beide Augen,
aber vor dem Kampf verband er sich eines. Das H"angen am Baum steht auch im
Zusammenhang mit der Kreuzigung von Jesus Christus. Christus hing drei Tage,
Odin 9 Tage, ein vielfaches von 3 (3 ist eine immer wiederkehrende Zahl in
den Religionen).  Beide waren jemandem geweiht, Odin sich selbst und Christus
der Menschheit. Beide stiessen im Augenblick der Wahrheit einen Schrei aus.
Daher kam es vor, dass die Kreuzigung von Christus mit dem H"angen von Odin
vermischt wurde.\\

Bevor Wotan (Odin) die Vormachtstellung unter den germanischen G"ottern
hatte, war ein anderer Gott mit dem Namen T\'{y}r der H"ochste. Wie gross
damals die G"otterzunft war kann heute nicht mehr gesagt werden. In den
Runen, die magische Schrift der Germanen, wird ersichtlich, dass Wotan als
Symbol nicht aufgef"uhrt wird sonder nur das 't' (teiwaz) f"ur den Himmelgott
(T\'{y}r) und das 'n' f"ur den Fruchtheitsgott, auch die beiden
G"ottergruppen (Asen, Wanen) haben ein eigenes Symbol. Der Himmelgott wird
auch in anderen Religionen benannt, z.B. irish Oll�thir, lat. Juppiter,
griech. Zeus pater, was alles soviel wie \glqq Vater Himmel\grqq\ bedeutet.
Daher muss das Allvater in der Edda nicht unbedingt auf den christlichen
Einfluss hinweisen, sondern, dass dieser Himmelgott der h"ochste Gott der
Indogermanen war. T\'{y}r ist wie Wotan ein Kriegsgott. Der Name des
T{\'{y}rs ist heute noch in unserem Wochentag Dienstag (aengl. Tiwesdaeg,
alem. Zu?stac; T\'{y}r = ahd. Ziu) bezeugt. Das verschwinden der Gottheit
kann auch dadurch erkl"art werden, dass die G"otter in der polytheistischen
Systemen nicht unsterblich sind und dieser T\'{y}r sp"ater im Gott Odin
wieder auslebte oder ersetzt wurde, also eine Inkarnation.\\
Die Herleitung sprachlich zu diesem Namen kommt von idg. {sl *deiwos} was
\glqq Gott\grqq\ bedeutet, es ist eine ethymologische Ableitung von {\sl *diw}
\glqq Himmel\grqq, daher dann {\sl *deiwos} \glqq der Himmlische\grqq, im
Gegensatz dazu im lat. {\sl Homo} \glqq der Irdische\grqq\ gegen"uber lat.
{\sl deus} als Ableitung von {\sl *deiwos}. Im Germanischen heisst dies dann
{\sl *teiwaz} bzw. {\sl *gumo}. Dieser Ausdruck {\sl *teiwaz} wird in der
Voluspa als Tivar benannt also in Kurzform T\'{y}r. Ein Beiname Odins
wei?t noch auf die Verwandschaftlichkeit mit T\'{y}r, in hanga-t\'{y}r
\glqq H"ange-Gott\grqq.\\

In den alten Religionen sehen wir, wie stark unser Leben mit solchen
Vorstellungen verkn"upft war und immer noch ist. In unserer heutigen
Gesellschaft wird die Religion immer mehr in den Hintergrund gedr"uckt, da
die technische Wissenschaft angeblich bewiesen hat, dass alles durch die
Evolution entstanden ist. Aber die alten Schriften k"onnen auch andere
Informationen beinhalten, die sich nicht nur auf die Entstehung der Erde
beziehen. Aus diesen B"ucher kann auch gelesen werden, wie die Menschen
gelebt haben, was sie f"ur richtig hielten und was nicht. Wenn die Moslems
kein Schweinefleisch essen muss das keinen religi"osen Hintergrund haben,
denn man weiss, dass die Schweine viele Krankheiten "ubertragen k"onnen. Dies
kann eine Begr"undung f"ur das Verbot sein.

%%-------------------  Ende des Textes  ------------------%%

\end{document}
