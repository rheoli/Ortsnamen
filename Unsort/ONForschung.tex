% Thema: Ortsnamenforschung
% Titel: Ortsnamenforschung in Niedersachsen
% Entnommen aus HTML-Seite
% Datum: 24.11.1996

%%%%%  Hauptdefinitionen f"ur Texte  %%%%%%
\documentclass[10pt]{article}
\usepackage{latexsym}   % ?????
\usepackage{german}     % Deutsche Sonderzeichen und Befehle
\usepackage{a4}         % A4 Ausrichtungen

%% \pagestyle{headings}    % Seitenaufbau mit Kapitelnamen als Kopfzeile
\pagestyle{myheadings}
\markright{\protect\rule[-5pt]{\linewidth}{1pt}\hspace{-\linewidth}%
{\protect\large Geschichte\ \  Keltische Ortsnamen in der Schweiz}%
}

\setlength{\evensidemargin}{-1.2cm}   % Breite f"ur l"ocherseitem Rand einstellen
\setlength{\oddsidemargin}{-0.5cm}    % Breite von gegenseitigem Rand

\setlength{\topmargin}{-1cm}          % H"ohe von oberen Rand einstellen

\setlength{\textwidth}{17.5cm}        % Textfensterbreite
\setlength{\textheight}{23.5cm}       % Textfensterh"ohe
% \setlength{\mathindent}{35mm}       % Formeln Abstand vom Rand

\setlength{\parindent}{0em}           % Absatzeinr"uckung

\renewcommand{\baselinestretch}{1.2}  % ???

%%%%%%%%%%%%%%%%%%%%%%%  Dokument Beginn  %%%%%%%%%%%%%%%%%%%
\begin{document}

%%================== Beginn der Texte  =============%%

\setlength{\parskip}{0.6em}           % Absatzabst"ande nur f"ur Texte

\section*{Ortsnamenforschung in Niedersachsen - Theorie und Praxis}
Bereits seit einigen Semestern findet sich donnerstags eine kleine Gruppe von
Interessierten im Sprachwissenschaftlichen Seminar in der Humboldtallee zu
Veranstaltungen zusammen, die die Aufarbeitung der Ortsnamen Niedersachsens
zum Inhalt haben. Unter der Leitung von Prof. J"urgen Udolph versuchen
Studenten der Sprachwissenschaft, interessierte Laien, Vertreter der
Geographie und der Geschichtswissenschaft, den Siedlungsnamen Niedersachsens
ihre Geheimnis zu entlocken: eine oft leichte, gelegentlich aber auch
schwierige und bei manchen Namen bis heute nicht gel"oste Aufgabe.\\

Der Ort f"ur diese Zusammenkunft ist gut gew"ahlt: das Seminar ist Teil des
Geb"audekomplexes, das den Namen Jacob Grimms tr"agt. Dieser hatte in der
Aufarbeitung der Ortsnamen eine wichtige Aufgabe gesehen: es gebe, so meint
er, \glqq ein lebendigeres Zeugnis "uber die V"olker, als Knochen, Waffen und
Gr"aber, und das sind ihre Sprachen\grqq, und an anderer Stelle: \glqq Ohne
die Eigennamen w"urde in ganzen fr"uhen Jahrhunderten jede Quelle der
deutschen Sprache versiegt sein, ja die "altesten Zeugnisse, die wir
"uberhaupt f"ur diese aufzuweisen haben, beruhen gerade in ihnen ... eben
deshalb verbreitet ihre Ergr"undung Licht �ber die Sprache, Sitte und
Geschichte unserer Vorfahren\grqq.\\

In den bisherigen Uebungen wurden die Kreise G"ottingen, Goslar, Holzminden,
Northeim und Osterode fast vollst"andig behandelt, f"ur den Kreis Hildesheim
reichte das letzte Semester allerdings nicht ganz und von dem Kreis
Wolfenb"uttel konnte nur ein kleiner Teil bearbeitet werden. Im Zusammenhang
mit dem Projekt sind erste Magisterarbeiten entstanden; die weitere
Aufarbeitung ist geplant. Dabei sollen auch die slavischen Namen am Ostrand
Niedersachsens einbezogen werden.\\

Wie "uberall auf der Welt sind die Namen auch in S"udniedersachsen, das nun
schon fast komplett durchgesehen wurde und von dem im folgenden vor allem die
Rede sein soll, historisch geschichtet. M"uhelos erkennt auch ein
sprachwissenschaftlicher Laie, da� sich hinter Mackenrode, Osterode,
Barterode, Volkerode Rodungsorte verbergen - allerdings ist die Analyse der
Erstglieder der Namen schon schwieriger. Sprachlich Interessierte werden
wissen, da� bis zum 16./17. Jh. und dar"uber hinaus in S"udniedersachsen fast
nur Niederdeutsch gesprochen wurde. Das hat im Namenschatz nat"urlich seine
Spuren hinterlassen: die Orte hei�en Lutterbeck und nicht Lauterbach,
Nienstedt und nicht Neustadt, Oldenrode und nicht Altenrode, Holtensen und
nicht Holzhausen. Allerdings hat die Verhochdeutschung die alten Verh"altnisse
verwischt, Mengershausen, Sieboldshausen, Lemshausen machen auf den ersten
Eindruck einen hochdeutschen Eindruck (Haus), aber alte Belege wie
Mengershusen, Siboldeshusen, Lemmenshusen zeigen die niederdeutsche Form
(hus) ebenso wie h"aufig noch die mundartliche Aussprache.\\

Die Analyse eines Namens ist jedoch am schwierigsten in F"allen, in denen
keinerlei Anklang an ein hochdeutsches oder niederdeutsches Wort vorliegt.
Was bedeuten Dransfeld, Dassel, Dr"uber, Harste, Schlarpe, Uslar, Katlenburg,
Echte, Kalefeld? Woher stammen die Namen, welcher Sprache sind sie zuzuordnen,
wie alt sind sie?\\

Der Weg zu einer L"osung eines Ortsnamens f"uhrt grunds"atzlich "uber die
Sammlung m"oglichst aller alten Belege. Diese gewinnt man aus gedruckten
Quellen, z.B. Urkundenb"uchern, Sammlungen von Verordnungen, Besitzt"umern,
aus Grenzbeschreibungen, privaten Quellen und anderen Texten. Die Durchsicht
f"uhrt f�r den Ort Echte z.B. zu folgender Belegreihe: 8./9. Jh. (Abschr.
12.~Jh.) Ethi, 973 in loco Ehte, um 979 Ehte, um 1024 in Hechti suum predium,
1191 (A. 14.Jh.) in Echte, 1210 in Echthe, 1218 in Echte, 1223 Ecthe, 1240 in
Hechte, 1273 in villa Echte. Aus dieser Reihe kann der Sprachwissenschafter
erste Erkenntnisse f"ur die Beurteilung des Namens gewinnen: zum einen
erweckt der erste Beleg mit seiner Folge Ethi Zweifel, zum anderen begegnet
zweimal ein sogenanntes \glqq prothetisches\grqq\ (vorangestelltes)
H- (Hechti, Hechte). Ber"ucksichtigt man diese und "ahnliche Abweichungen,
Schreibeinfl"usse, Varianten, auch offensichtliche Verschreibungen usw., so
kann man im g"unstigsten Fall eine Grundform gewinnen, die die Basis f"ur die
Namendeutung abgibt. Im vorliegenden Fall ist es wohl ein Ansatz {\em *Ahtjo}
auszugehen, der ein Sternchen erh"alt, da diese Form nicht belegt ist,
sondern nur erschlossen wurde und als Arbeitshypothese dienen kann. Die
weitere Untersuchung bedient sich bew"ahrter Methoden der
historisch-vergleichenden Sprachwissenschaft: der Name wird hinsichtlich
seiner Lautung dahingehend gepr"uft, welche Entwicklungen sich im Vergleich
zu anderen und "ahnlichen W"ortern und Namen im Niederdeutschen,
Westgermanischen und gegenbenfalls noch "alteren Sprachstufen wahrscheinlich
machen lassen. In Norddeutschland ist immer zu fragen, ob er die Phase der
sogenannten germanischen Lautverschiebung durchlaufen hat. In unserem Fall
spricht das -h- in Echte (anstelle eines alten, vorgermanischen -k-) f"ur
diese Annahme. Trotz aller Bem"uhungen findet man f"ur Echte aber keinen
sicheren Anschlu� im Germanischen; es handelt sich zweifelsfrei um einen
hochaltert"umlichen Ortsnamen, dessen Namengebung in die Vorzeit hineinragt
und der der Forschung besondere Probleme bietet.\\

Die hier knapp skizzierte Methode erfordert den Umgang mit historischen
Quellen, mit W"orterb"uchern, Grammatiken und Texten nicht nur des
Niederdeutschen, Hochdeutschen und den verwandten germanischen Sprachen und
Dialekten, sondern auch - vor allem bei schwierigen Namen - Kenntnisse
indogermanistischer Zusammenh"ange.\\ 

Zu dieser mehr theoretischen und auf auf die Durchsicht von Publikationen
ausgerichteten Seite der Untersuchung kann aber eine praktisch orientierte
Ueberpr"ufung hinzukommen: der Blick auf die Oertlichkeit, die den Namen tr"agt.
Der Namenforscher nennt dieses \glqq Realprobe\grqq. Nach Abschlu� der
Untersuchung der Ortsnamen des Kreises Northeim entschloss man sich im
Sprachwissenschaftlichen Seminar im Mai 1994 zu einem ersten Versuch. Die
Exkursion f"uhrte zu den Orten B"uhle, Berka, Hollenstedt, Wetze, Dr"uber,
Hohnstedt, Echte, Opperhausen, Kuventhal, Dassel, Nienover, Wahmbeck,
Schoningen, Vahle, Tr"ogen und H"ockelheim. Nicht immer erf"ullten sich die
Hoffnungen, die Lage des Ortes, die Bod engestalt, -form oder -qualit"at
k"onne bei strittigen Punkten den rettenden Einfall bringen. In einigen
F"allen aber half die Besichtigung weiter. So war man sich bei der Beurteilung
des seltsamen Namens Dr"uber (ein Drunter, Hin"uber o."a. spielt mit
Sicherheit keine Rolle) schon im Seminar einig gewesen, da� der Name ein
-r-Element besitzen d"urfte (wie das benachbarte Iber, wie auch Letter,
Limmer und (Salz)gitter), da� von einer Grundform {\em *Drubira} auszugehen
sei und ein Element zugrunde liegen m"usse, da� verwandt sei mit griechisch
����� \glqq zerreiben, zerbr"ockeln\grqq, lettisch drubazas
\glqq Holzsplitter\grqq, drupu, drupt \glqq zerfallen, in Tr"ummer gehen\grqq,
dra�p�t \glqq zerbr"ockeln\grqq\ u.a.m.\\

Die Besichtigung des Ortes brachte die Best"atigung: er liegt auf einer
Landzunge an einer steilen B"oschung des Leinetals. Steht man am Ostrand des
Ortes, liegt der Abhang direkt vor den F"ussen des Betrachtenden. Dieser
"uberzeugende Fall ist nicht die Regel; jeder Name ist ein Problem f"ur sich,
Verallgemeinerungen sind zu vermeiden.\\

In einer zweiten Exkursion wurde im Juli 1994 der Kreis Holzminden mit den
Orten Vorwohle, Emmerborn, Heinade, Hellental, Braak, Schorborn, Negenborn,
Warbsen, Holenberg, Wickensen, Tuchtberg, Tuchtfeld, Halle, Kreipke, Linse,
Thran, Hehlen, Ovelg"onne, Br"okeln, Hohe, D"olme, Pegestorf, R"uhle, Polle
und Derental besucht. Die gr"osste Ueberraschung brachte der an sich nicht
sehr aufregende Ortsname Negenborn, 983-985 (Abschrift 15. Jh.) Nighunburni,
(1015-1036) (A. 13. Jh.) Niganbrunnun, (1155-1184) (A. 13. Jh.) filii H. de
Nigenborne usw., in dem neben Born \glqq Quelle, Born, Brunnen\grqq\ im
allgemeinen niederdeutsch nije, nige \glqq neu\grqq\  gesehen wird. Die
M"oglichkeit einer Verbindung mit negen, nejen \glqq neun\grqq\ wurde
(befanden sich hier einmal so viele Quellen?) f"ur unwahrscheinlich gehalten.
Umso "uberraschender war der Anblick eines Wirtshauses an der Hauptstrasse
mit dem Namen Neunbrunnen und die Abbildung zahlreicher Brunnen an der
Hausfassade. Der Versuch, bei den Besitzern etwas "uber den Namen zu erfahren,
blieb ohne Ergebnis; die T"uren waren verschlossen. So wird man von G"ottingen
aus versuchen, der Sache auf den Grundzu gehen.\\

Das Ziel der Unternehmung - ein Historisches Ortsnamenbuch f"ur Niedersachsen -
liegt mit seiner Fertigstellung noch in weiter Ferne. Bis jetzt ist nur der
s"udliche Zipfel Niedersachsens fl"achendeckend behandelt worden. Wichtige
und dicht besiedelte Gebiete um Braunschweig, Hildesheim, Hannover, Osnabr"uck
werden noch viel Arbeit erfordern, zumal die Ortsnamenforschung in
Niedersachsen bisher eher von wissenschaftlichen Laien (so etwa f"ur den
Kreis Diepholz) betrieben wurde. Erst in diesen Tagen ist die erste fundierte
Abhandlung "uber die Siedlungsnamen eines nieders"achsischen Kreises
zug"anglich geworden: es ist das Buch des schleswig-holsteinischen (!)
Ortsnamenspezialisten Wolfgang Laur "uber die Ortsnamen des Kreises Schaumburg
(Rinteln 1993).\\

Dabei hat bereits der auch in G"ottingen lehrende Jacob Grimm von der Bedeutung
der Siedlungsnamen gewusst: auf seine und seines Bruders Anregung stellte die
Berliner Akademie im Jahr 1849 eine Preisaufgabe, die darin bestand, eine
m"oglichst vollst"andige Zusammenstellung der altdeutschen Personen- und
Ortsnamen bis zum Jahre 1100 vorzulegen. Zwar erf"ullte niemand diese Aufgabe
vollst"andig, aber der von Ernst F"orstemann eingereichte \glqq Entwurf\grqq\
wurde zur Grundlage des bisher immer noch nicht ersetzten und mehr als 100
Jahre alten Altdeutschen Namenbuchs. Nun wird in G"ottingen der Versuch
gemacht, wenigstens f"ur Niedersachsen eine modernen Anspr"uchen gen"ugende
Zusammenstellung der Siedlungsnamen vorzunehmen. Das Sommersemester 1995 ist
den Ortsnamen des Kreises Hameln-Pyrmont gewidmet; wie immer wird sich
donnerstags eine kleine Gruppe von Interessierten im Sprachwissenschaftlichen
Seminar versammeln und die R"atsel der Siedlungsnamen zu l"osen versuchen.
Zuvor wird man sich aber am 1.~April zu einer weiteren Exkursion, dieses Mal
in den Kreis Hildesheim, treffen.\\

von Prof. Dr. J"urgen Udolph

%%-------------------  Ende des Textes  ------------------%%

\end{document}